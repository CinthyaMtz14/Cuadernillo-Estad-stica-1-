%Example of use of oxmathproblems latex class for problem sheets
%(un)comment this line to enable/disable output of any solutions in the file
%\printanswers
\documentclass{../oxmathproblems}
\usepackage{blindtext}
\usepackage{hyperref}
\usepackage{geometry}
%define the page header/title info
\course{ITAM - Estadistica 1}
\oxfordterm{Cuadernillo de Ejercicios}
\sheetnumber{1}
\sheettitle{author: Cinthya Paulina Martínez Andrade
}
\extrawidth{2cm}

\begin{document}

\begin{questions}

\miquestion \textbf{Datos cualitativos y distribuciones de frecuencia}

Identifica cada una de las siguientes variables como cualitativas o cuantitativas y su escala de medición. 
\begin{itemize}

\item a) El tipo de uso más frecuente de su horno de microondas (recalentar, descongelar, calentar)

\item b) El número de consumidores que se niegan a contestar una encuesta por teléfono 

\item c) La puerta escogida por un ratón en un experimento de laboratorio (A, B o C)

\item d) El tiempo ganador para un caballo que corre en el Derby de Kentucky 

\end{itemize}




\miquestion Se realizó una encuesta a los alumnos del ITAM, respecto a la marca de cigarros que consumen con frecuencia. Se obtuvieron los siguientes datos: 
\textit{Camel, camel, delicado, delicado, camel, marlboro, marlboro, camel, delicado, delicado}

\text{Realice la tabla de distribución de frecuencia}






\miquestion \textbf{Distribución de frecuencias y diagrama de puntos}
Se enumeran los colores de 21 dulces: 

\textit{Café, rojo, amarillo, café, anaranjado, amarillo, verde, rojo, anaranjado, azul, azul, café, verde, verde, azu , café, azul, café, azul, café, anaranjado}

Construya la tabla de distribución de frecuencia para la variable cualitativa y su respectivo diagrama de puntos y gráfica de barras

\miquestion \textbf{Diagrama de tallo y hoja variables discretas}
Se tiene los siguientes precios en dolares de zapatos deportivos para el primer trimestre del 2022:

\textit{90, 65, 75, 70, 70, 68, 70, 70, 60, 68, 70, 74, 65, 75, 70, 40, 70, 95, 65}

Construya un diagrama de tallo y hoja para estos datos. Usted proponga la escala del tallo y las hojas.

\miquestion \textbf{Variable aleatoria continua}
El responsable de control de calidad de "Mr. Tuky Hot Dog" debe verificar el peso de las bolsas de 2.27 kg. de salchicha. Para cumplir con su tarea sin verificar cada bolsa que sale de la planta, el responsable muestrea diariamente algunas bolsas, pesa el contenido y extrae una conclusión sobre el peso promedio de las bolsas que salen de esa planta ese día. Se presenta el peso de 15 bolsas seleccionadas en un día: 

\textit{2.15, 2.18, 2.19, 2.19, 2.19, 2.19, 2.21, 2.27, 2.27, 2.27, 2.27, 2.27, 2.27, 2.30, 2.33}

Calcular: 
\begin{itemize}
\item a) Tabla de distribución de frecuencias
\item b) Diagrama de tallo y hoja
\item c) Diagrama de puntos 
\item d) Histograma 
\end{itemize}




\textbf{Distribuciones de probabilidad} 
\miquestion Sea x una distribución de probabilidad para una variable aleatoria discreta, se tiene los siguientes datos: 


\begin{enumerate}
\item 
\begin{tabular}{| c | c |}
\hline
x & p(x) \\ \hline
0 & $\frac{1}{4}$ \\ \hline
1 &$\frac{1}{2}$\\ \hline
2 & $\frac{1}{4}$ \\ \hline
\end{tabular}

\item  
\begin{tabular}{| c | c |}
\hline
x & p(x) \\ \hline
0 & 0.10 \\ \hline
1 &  0.40 \\ \hline
2 & 0.20 \\ \hline
3 & 0.15 \\ \hline
4 & 0.10 \\ \hline
5 & 0.05 \\ \hline
\end{tabular}
\end{enumerate}

\begin{itemize}
\item Para cada uno de los incisos: determine el valor esperado de x, la media y la desviación estándar. 
\end{itemize}


\miquestion En una lotería realizada a beneficio de la institución local de caridad, se han de vender 8000 boletos a $10$ cada uno. El premio es un automóvil con valor de $24 000$. Si usted compra dos boletos, ¿cuál es su ganancia esperada?

\miquestion Sea x una distribución de probabilidad para una variable aleatoria discreta, se tiene los siguientes datos: 
\begin{tabular}{| c | c |}
\hline
x & p(x) \\ \hline
0 & 0.10 \\ \hline
1 &  0.3 \\ \hline
2 & 0.4 \\ \hline
3 & 0.1 \\ \hline
4 & $?$ \\ \hline
5 & 0.05 \\ \hline
\end{tabular}
\begin{itemize}
\item Encuentre la $p(x=4)$
\item Determine el valor esperado de x, la media y la desviación estándar. 
\end{itemize}


\miquestion Un jugador de fútbol anota más goles a corta distancia. Sea  x el número de goles anotados. Se tiene la siguiente distribución de probabilidades: 



\text{Determine, cuál es la puntuación esperada del jugador} 


\miquestion \textbf{Repaso de Integrales (Parte (1))}
\begin{itemize}
\item a) $\int  \frac{1}{x^2 \sqrt[5]{x^2}} dx$ 
\item b) 
$\int \mathrm{e}^x+2x^3 dx $

\item b) $\int (x+2)^3 dx  $ 

\item c) $ \int 5^x dx  $ 
\item d) $ \int 2^x \cdot 5^2 dx $ 
\item e) $ \int \frac{x+1}{x} dx $ 
\item f) $ \int x^5 dx $ 
\item g) $ \int \mathrm{e}^x dx $ 
\item h) $ \int \frac{x^2}{\sqrt{x}} dx $ 
\item i) $ \int  \sqrt{x+1} dx $ 
\item j) $ \int  (\sqrt{x} + a) dx $ 
\item k) $ \int x + \sqrt{x} dx $ 
\item l) $ \int x  dx $ 
\item m) $ \int \frac{1}{x} dx  $
\item n) $ \int (\frac{1}{x} + 1) dx $
\item o) $ \int (\frac{3x^2}{x} + 1 ) dx $ 
\item p) $ \int (3x^2 + 1 ) dx $ 
\item q) $ \int (2x - \mathrm{e}^x ) dx $ 
\item r) $ \int (2x^3 - \mathrm{e}^x ) dx $ 
\item s) $ \int 3x^4 dx $ 
\item t)  $ \int 2x^7 dx $ 
\item u)  $ \int \frac{1}{x^3} dx $ 
\item v) $ \int \frac{3}{t^5} dt $ 
\item w) $ \int 5u^{\frac{3}{2}} du $ 
\item x) $ \int  \frac{3}{\sqrt{y}} dx $ 
\item y) $ \int (4x^3 + x^2) dx $ 
\item z) $ \int (3u^5 - 2u^3) du $ 

\end{itemize}



\miquestion \textbf{Repaso de Integrales (Parte (2))}
\begin{itemize}
\item a) $ \int y^3(2y^2 - 3) dx $ 
\item b) $ \int x^4(5 - x^2) dx $ 
\item c) $ \int (3 -2t + t^2) dx $ 
\item d) $ \int \sqrt{x}(x+1) dx $ 
\end{itemize}

\miquestion \textbf{Repaso de Integrales (Integración por partes)}
\begin{itemize}
\item a) $ \int \ln{^2}x dx $ 
\item b) $ \int (x^3 + 5x^2 - 2) dx $ 
\item c) 
\end{itemize}


\textbf{DISTRIBUCIONES DE PROBABILIDAD}

\emph{ Distribuciones de probabilidad para variables aleatorias continuas}


\text{Las}  \textbf{variables aleatorias continuas, } \text{ por ejemplo estaturas y pesos, lapso de vida útil de un
producto en particular o un error experimental de
laboratorio, pueden tomar los numeros infinitamente correspondientes a un intervalo de una recta.} 











\end{questions}




\textbf{Bibliografía}
Mendenhall, W. (2006). Introducción a la probabilidad y Estadística (Vol. 13). Cengage Learning.
Aguirre, V. A. B. A. (2006). Fundamentos de Probabilidad y Estadística (2 ed.). Jit Press.

\end{document}
